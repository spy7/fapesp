\documentclass[12pt]{report}

\usepackage[a4paper]{geometry}
\usepackage[utf8]{inputenc}
\usepackage[english, portuguese]{babel}
\usepackage[T1]{fontenc}
\usepackage{fancyhdr}
\usepackage{setspace}
\usepackage{indentfirst}
\usepackage{titlesec}

\title{Plataforma de big data público de mapeamento de indícios de irregularidade nas compras públicas, comparação municipal e efetividade dos gastos públicos}
\author{Carlos André Sanches de Souza, Ms.}
\date{\today}

\makeatletter
	\let\titulo\@title
	\let\autor\@author
	\let\hoje\@date
\makeatother

\newcommand{\responsavel}{Lucas Madureira dos Anjos}
\newcommand{\instituicao}{Gedanken Desenvolvimento e Licenciamento de Programas de Computador Ltda.}
\newcommand{\instituicaoSigla}{Gedanken}
\newcommand{\fomento}{Fundação de Amparo à Pesquisa do Estado de São Paulo}
\newcommand{\fomentoSigla}{FAPESP}
\newcommand{\tipoRelatorio}{Anual}
\newcommand{\modalidadeProjeto}{Treinamento Técnico TT-5}
\newcommand{\numeroProjeto}{2019/25945-4} 
\newcommand{\periodoVigencia}{01/05/2020 a 30/09/2021}
\newcommand{\periodoRelatorio}{01/05/2020 a 20/09/2020}
\newcommand{\cidade}{Ribeirão Preto}

\def\maketitle{
	\begin{titlepage}
		\begin{center}
			\singlespace
			{\large\MakeUppercase\instituicao}\\[3cm]
			\begin{spacing}{1.3}
				{\Large\textbf\titulo}\\[0.5cm]
			\end{spacing}
			\rule{\linewidth}{0.4mm}\\[0.3cm]
			Relatório Científico {\tipoRelatorio} do projeto na modalidade
			{\modalidadeProjeto}, fomentado pela \fomento.\\[0.15cm]
			\rule{\linewidth}{0.4mm}\\[0.5cm]
			Projeto \fomentoSigla : \numeroProjeto\\[0.5cm]
			Período de vigência do projeto: \periodoVigencia\\
			Período de cobertura do relatório: \periodoRelatorio\\[2cm]
			Bolsista: \textit{\autor}\\
			Responsável: \textit{\responsavel}\\
			\vspace*{\fill}
			{\cidade}, \hoje
		\end{center}
	\end{titlepage}
}

\def\abstract{
	\begin{center}
	\bfseries
	\abstractname
	\end{center}
	\thispagestyle{plain}
}

\fancypagestyle{plain}{
	\fancyhf{}
	\renewcommand\headrulewidth{0pt}
	\fancyfoot[R]{\thepage}
}

\titleformat{\chapter}[hang] 
{\normalfont\LARGE\bfseries\singlespace}{\thechapter.}{4mm}{} 

\titlespacing*{\chapter}{0pt}{0pt}{40pt}

\linespread{1.5}

\begin{document}

\pagenumbering{roman}

\maketitle

\chapter{Resumo do projeto proposto}
\label{chp:resumoProj} 
\pagenumbering{arabic}
O objetivo desta bolsa de Treinamento Técnico TT-5 é pesquisar e desenvolver, sob supervisão e treinamento de pesquisador \responsavel , novos robôs de mapeamento de indício de irregularidade com cruzamento dos dados já presentes no banco de dados e dos novos dados extraídos. A bolsa também tem como objetivo o desenvolvimento de algoritmos de correlação dos dados, a organização dos novos dados extraídos para o banco de dados e a criação de APIs para serem consumidas nas pesquisas e nos desenvolvimento de novos robôs de mapeamento de irregularidade. Terá como responsabilidade também a criação de redes sociais (neurais) para identificar relacionamento entre pessoas e empresa, além do desenvolvimento de novos módulos de efetividade dos gastos públicos e de tratamento dos indícios de irregularidade mapeados.

\chapter{Realizações do período}
\label{chp:realizacoes}
Aqui começa o segundo capítulo com as realizações no período.

\chapter{Descrição e avaliação do Apoio Institucional recebido no período}
\label{chp:apoioInst}
A \fomento\ financiou até o momento 4 meses do projeto, previsto para 17 meses, com a Bolsa Concedida como Itens Orçamentários em Auxílios (BCO) de Treinamento Técnico V (TT-V). O apoio financeiro da \fomentoSigla\ tem sido de considerável importância para a realização do projeto de pesquisa.

A empresa \instituicaoSigla\ acolhe a pesquisa e disponibiliza as condições necessárias para
sua realização com infraestrutura, algoritmos, acesso a Internet, dados coletados e orientação por parte do pesquisador \responsavel , responsável pela pesquisa.

\chapter{Plano de atividades para o próximo período}
\label{chp:planoAtiv}
Aqui começa o terceiro capítulo com a descrição da participação em eventos científicos.

\bibliographystyle{abntex2-num}
\bibliography{bibliografia}

\end{document}
