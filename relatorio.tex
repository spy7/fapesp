\documentclass[12pt]{report}

\usepackage[a4paper]{geometry}
\usepackage[utf8]{inputenc}
\usepackage[english, portuguese]{babel}
\usepackage[T1]{fontenc}
\usepackage{indentfirst}
\usepackage{setspace}
\usepackage{titlesec}
\usepackage{tikz}
\usepackage{tikz-3dplot}
\usepackage[shading]{moeptikz}
\usepackage[format=plain, labelfont=it, textfont=it]{caption}
\usepackage{etoolbox}
\apptocmd{\sloppy}{\hbadness 10000\relax}{}{}
\usetikzlibrary{calc, positioning, shapes, shapes.geometric}

% Variables
\title{Plataforma de big data público de mapeamento de indícios de irregularidade nas compras públicas, comparação municipal e efetividade dos gastos públicos}
\author{Carlos André Sanches de Souza}

\makeatletter
	\let\titulo\@title
	\let\autor\@author
\makeatother

\newcommand{\responsavel}{Lucas Madureira dos Anjos}
\newcommand{\instituicao}{Gedanken Desenvolvimento e Licenciamento de Programas de Computador Ltda.}
\newcommand{\instituicaoSigla}{Gedanken}
\newcommand{\fomento}{Fundação de Amparo à Pesquisa do Estado de São Paulo}
\newcommand{\fomentoSigla}{FAPESP}
\newcommand{\relatorio}{Relatório Individual Sintético}
\newcommand{\projetoModalidade}{Treinamento Técnico}
\newcommand{\projetoNivel}{V}
\newcommand{\projetoSigla}{TT-V}
\newcommand{\projetoNumero}{2019/25945-4}
\newcommand{\projetoVinculado}{2019/00804-9}
\newcommand{\periodoVigencia}{01/05/2020 a 30/09/2021}
\newcommand{\periodoRelatorio}{01 de maio de 2020 a 20 de setembro de 2020}
\newcommand{\cidade}{Ribeirão Preto}
\addto{\captionsportuguese}{%
  \renewcommand{\bibname}{Referências}
}

% Settings
\def\maketitle{
	\begin{center}
		{\Large \relatorio\ de \projetoModalidade}\\[0.5cm]
		{\large Bolsista: \autor}\\
		{\large Coordenador: \responsavel}
	\end{center}
}

\usepackage{etoolbox} 
\makeatletter 
\patchcmd{\chapter}{\if@openright\cleardoublepage\else\clearpage\fi}{}{}{} 
\makeatother 

\titleformat{\chapter}[hang] 
{\normalfont\Large\bfseries\singlespace}{\thechapter.}{4mm}{} 

\titleformat{\section}[hang] 
{\normalfont\large\bfseries\singlespace}{\thesection.}{4mm}{} 

\titlespacing*{\chapter}{0pt}{15pt}{15pt}

\linespread{1.5}

% Image symbols
\makeatletter
\pgfkeys{/pgf/.cd,
  parallelepiped offset x/.initial=2mm,
  parallelepiped offset y/.initial=2mm
}
\pgfdeclareshape{parallelepiped}
{
  \inheritsavedanchors[from=rectangle] % this is nearly a rectangle
  \inheritanchorborder[from=rectangle]
  \inheritanchor[from=rectangle]{north}
  \inheritanchor[from=rectangle]{north west}
  \inheritanchor[from=rectangle]{north east}
  \inheritanchor[from=rectangle]{center}
  \inheritanchor[from=rectangle]{west}
  \inheritanchor[from=rectangle]{east}
  \inheritanchor[from=rectangle]{mid}
  \inheritanchor[from=rectangle]{mid west}
  \inheritanchor[from=rectangle]{mid east}
  \inheritanchor[from=rectangle]{base}
  \inheritanchor[from=rectangle]{base west}
  \inheritanchor[from=rectangle]{base east}
  \inheritanchor[from=rectangle]{south}
  \inheritanchor[from=rectangle]{south west}
  \inheritanchor[from=rectangle]{south east}
  \backgroundpath{
    % store lower right in xa/ya and upper right in xb/yb
    \southwest \pgf@xa=\pgf@x \pgf@ya=\pgf@y
    \northeast \pgf@xb=\pgf@x \pgf@yb=\pgf@y
    \pgfmathsetlength\pgfutil@tempdima{\pgfkeysvalueof{/pgf/parallelepiped
      offset x}}
    \pgfmathsetlength\pgfutil@tempdimb{\pgfkeysvalueof{/pgf/parallelepiped
      offset y}}
    \def\ppd@offset{\pgfpoint{\pgfutil@tempdima}{\pgfutil@tempdimb}}
    \pgfpathmoveto{\pgfqpoint{\pgf@xa}{\pgf@ya}}
    \pgfpathlineto{\pgfqpoint{\pgf@xb}{\pgf@ya}}
    \pgfpathlineto{\pgfqpoint{\pgf@xb}{\pgf@yb}}
    \pgfpathlineto{\pgfqpoint{\pgf@xa}{\pgf@yb}}
    \pgfpathclose
    \pgfpathmoveto{\pgfqpoint{\pgf@xb}{\pgf@ya}}
    \pgfpathlineto{\pgfpointadd{\pgfpoint{\pgf@xb}{\pgf@ya}}{\ppd@offset}}
    \pgfpathlineto{\pgfpointadd{\pgfpoint{\pgf@xb}{\pgf@yb}}{\ppd@offset}}
    \pgfpathlineto{\pgfpointadd{\pgfpoint{\pgf@xa}{\pgf@yb}}{\ppd@offset}}
    \pgfpathlineto{\pgfqpoint{\pgf@xa}{\pgf@yb}}
    \pgfpathmoveto{\pgfqpoint{\pgf@xb}{\pgf@yb}}
    \pgfpathlineto{\pgfpointadd{\pgfpoint{\pgf@xb}{\pgf@yb}}{\ppd@offset}}
  }
}
\makeatother

\tikzset{
  ports/.style={
    line width=0.3pt,
    top color=gray!20,
    bottom color=gray!80
  },
  server/.style={
    parallelepiped,
    fill=white,
    draw,
    minimum width=0.35cm,
    minimum height=0.75cm,
    parallelepiped offset x=2.5mm,
    parallelepiped offset y=2mm,
    path picture={
      \draw[top color=gray!5,bottom color=gray!40]
      (path picture bounding box.south west) rectangle 
      (path picture bounding box.north east);
      \coordinate (A-center) at ($(path picture bounding box.center)!0!(path
        picture bounding box.south)$);
      \coordinate (A-east) at ([xshift=-0.525cm]path picture bounding box.east);
      \draw[ports]([yshift=0.1cm]$(A-east)!-0.065!(A-center)$)
        rectangle +(0.2,0.065);
      \draw[ports]([yshift=0.01cm]$(A-east)!0.06!(A-center)$)
        rectangle +(0.15,0.05);
      \fill[black]([yshift=-0.35cm]$(A-east)!-0.1!(A-center)$)
        rectangle +(0.235,0.0175);
      \fill[black]([yshift=-0.385cm]$(A-east)!-0.1!(A-center)$)
        rectangle +(0.235,0.0175);
      \fill[black]([yshift=-0.42cm]$(A-east)!-0.1!(A-center)$)
        rectangle +(0.235,0.0175);
    }  
  },
  database/.style={
    cylinder,
    font=\scriptsize,
    shape border rotate=90,
    minimum height=1.5cm,  
    text width=1.4cm, 
    align=center,      
    aspect=0.25,
    draw
  },
  api/.style={
    rectangle,
    font=\scriptsize,
    text width=0.3cm,
    minimum width=0.6cm, 
    minimum height=2cm,  
    align=center,    
    draw
  },
  robot/.style={
    rectangle,
    minimum width=12mm,
    minimum height=11mm,
    path picture={
    		\draw[rounded corners] (-4mm,-5mm) rectangle (4mm,1mm);
		\draw (2mm,-2mm) circle (1mm);
		\draw (-2mm,-2mm) circle (1mm);		
		\draw (0,3mm) circle (1mm);
		\draw (0,2mm) -- (0,1mm);
		\draw (-4mm,-0.8mm) arc[radius=1.5mm,start angle=100,end angle=260];
		\draw (4mm,-0.8mm) arc[radius=1.5mm,start angle=80,end angle=-80];
    }
  },
  docs/.style={
    rectangle,
    minimum width=2cm,
    minimum height=2cm,
    path picture={
      \draw[line width=0.9mm] (-2.25mm,7.25mm) --++ (0,1.25mm) --++ (6.5mm,0) --++ (2.5mm,-2.5mm) --++ (0,-10mm) --++ (-1.25mm,0);
      \fill[black] (3.75mm,8.5mm) --++ (3mm,-3mm) --++ (-3mm,0) -- cycle;
      \draw[line width=0.9mm] (-4.5mm,5mm) --++ (0,1.25mm) --++ (6.5mm,0) --++ (2.5mm,-2.5mm) --++ (0,-10mm) --++ (-1.25mm,0);
      \fill[black] (1.5mm,6.25mm) --++ (3mm,-3mm) --++ (-3mm,0) -- cycle;
      \draw[line width=0.9mm] (-6.75mm,4mm) --++ (6.5mm,0) --++ (2.5mm,-2.5mm) --++ (0,-10mm) --++ (-9mm,0) -- cycle;
      \fill[black] (-0.75mm,4mm) --++ (3mm,-3mm) --++ (-3mm,0) -- cycle;
      \foreach \n in {0.25mm, -1.375mm, -3mm, -4.625mm, -6.25mm}{
    	    \draw[line width=0.75mm, line cap=round] (-5.25mm,\n) -- (0.75mm,\n);
      }
    }
  }
}


% Document
\begin{document}

\begin{tikzpicture}
\node[docs] (d1) {};
\end{tikzpicture}


\maketitle

\chapter{Introdução}
\label{chp:introducao} 
Este documento é o relatório individual sintético do bolsista \autor\ referente ao \projetoModalidade\ (nível \projetoNivel , processo número \projetoNumero) realizado de \periodoRelatorio , sob a coordenação do pesquisador \responsavel. O \projetoModalidade\ foi realizado em apoio ao projeto de pesquisa "\titulo"\ (processo número \projetoVinculado), financiado pela \fomentoSigla.

-- O bolsista participou do desenvolvimento de um novo software para a segmentação dos vasos sangüíneos de imagens de retina, escrito em C++; cuidou da manutenção e suporte de uma versão anterior do software, que havia sido feita em linguagem Matlab durante seu Mestrado; e colaborou em trabalhos com outros pesquisadores, principalmente o Prof. Dr. Herbert F. Jelinek, da australiana Charles Sturt University e o oftalmologista Marcelo B. M. Mendonça, pós-graduando da UFPE, que têm interesse no uso dos softwares e ajudaram a validar os métodos desenvolvidos.

Durante seu Mestrado, o bolsista tinha produzido avanços significativos no método de segmentação de vasos sangüíneos baseado em wavelets e classificadores estatísticos [4,6]. Um protótipo de software de segmentação incluindo uma interface gráfica foi implementado na linguagem Matlab durante o Mestrado. A Bolsa de Treinamento Técnico apoiou a implementação de um novo software de segmentação de vasos sangüíneos, contando com métodos já consolidados de bom desempenho. O novo software foi escrito em C++, apresentando tempos menores de processamento e, ao contrário da versão anterior (que necessitava do uso do Matlab), pode ser usado sem a necessidade de licenças. Encontra-se disponível no portal do projeto (http://retina.iv.fapesp.br), sob a licença GPL.

A análise automatizada das imagens de retina, com especial atenção à estrutura vascular, permitirá uma triagem mais abrangente de pacientes diabéticos para a detecção da retinopatia diabética (RD). Com isso, espera-se aumentar a taxa de detecção precoce da doença, o que previne perdas visuais na grande maioria dos casos, tornando possıveis intervenções médicas [3, 7]. O software desenvolvido representa um esforço para a construção de uma ferramenta de auxılio diagnóstico na triagem para a RD, de grande importância para a saúde pública.

-- O objetivo desta bolsa de Treinamento Técnico TT-5 é pesquisar e desenvolver, sob supervisão e treinamento de pesquisador \responsavel , novos robôs de mapeamento de indício de irregularidade com cruzamento dos dados já presentes no banco de dados e dos novos dados extraídos. A bolsa também tem como objetivo o desenvolvimento de algoritmos de correlação dos dados, a organização dos novos dados extraídos para o banco de dados e a criação de APIs para serem consumidas nas pesquisas e nos desenvolvimento de novos robôs de mapeamento de irregularidade. Terá como responsabilidade também a criação de redes sociais (neurais) para identificar relacionamento entre pessoas e empresa, além do desenvolvimento de novos módulos de efetividade dos gastos públicos e de tratamento dos indícios de irregularidade mapeados.

\chapter{Atividades realizadas}
\label{chp:realizacoes}
O bolsista vem realizando os trabalhos conforme foram propostos inicialmente pelo programa. Ao mesmo tempo, o bolsista vem buscando novas metodologias para melhorar a qualidade das implementações, dos dados e da forma de trabalho que deverá resultar em sistemas mais sólidos e dados mais confiáveis.

\section{Tratamento de dados da Receita Federal}
Foram coletados dados públicos mais recentes pelo site da Receita Federal \cite{receita}. Os dados foram armazenados localmente em um banco de dados PostgreSQL \cite{postgresql} e novas tabelas foram geradas a partir das originais.

Dentre as tabelas geradas, podemos destacar a tabela de documentos, que reune em uma única tabela todos os CPFs e CNPJs encontrados nos dados públicos, permitindo consultas rápidas a todos os dados, e a tabela de sociedade, que traz o relacionamento entre empresas e sócios com as informações mais relevantes como data de entrada e percentual de participação.

Este tratamento envolvendo a criação de tabelas e índices trará um alto ganho de performance nas consultas realizadas pelo sistema atual, principalmente nas consultas realizadas pelos robôs de indentificação de irregularidades que precisam capturar e analisar dados de diversas fontes.

Todo o tratamento realizado está sendo documentado e armazenado na forma de scripts nas linguagens Python \cite{python} e SQL para que possa ser reproduzido futuramente com a disponibilização de novos dados públicos.

\section{Criação de API para os dados tratados}
Foi desenvolvida uma API \cite{api} utilizando o framework Django \cite{django} em conjunto com o framework de criação de APIs chamado Django Rest Framework \cite{drf}. Esta API carrega os dados públicos que já foram tratados e armazenados em banco de dados e os disponibilizam através de endpoints (endereços URL para cada serviço) específicos para serem acessados por outras aplicações clientes.

A utilização da API permite controlar o acesso aos dados através de autenticação e filtros de acordo com o tipo de usuário conectado. Os dados são analisados, calculados e agrupados para cada serviço oferecido e disponibilizados ao cliente no formato JSON \cite{json}.

O bolsista trabalha em contato com a equipe de desenvolvimento da aplicação que irá consumir os dados enviados pela API, fornecendo suporte e adaptando a saída de dados de acordo com a necessidade da aplicação.

% \vspace{\baselineskip}

\begin{figure}[h!]
\centering
\begin{tikzpicture}[>=stealth]
\node[database] (db1) at (0,0) {Banco de dados};
\node[server, right=3cm of db1, scale=1.5, label={[font=\scriptsize,align=center]south:servidor\\[-1mm]web}] (srv1) {};
\node[api, right=0.8cm of srv1] (api1) {A P I};
\node[notebook, right=5cm of api1,label={[font=\scriptsize,align=center]south:aplicação},yshift=-2mm] (nb1) {};
\draw[->] (db1) -- node[midway,above,font=\scriptsize]{dados tratados} (srv1);
\draw[<-] ($(api1)+(0.3,0)$) -- node[near start,above,font=\scriptsize]{requisição} ($(nb1)+(-0.5,0.2)$);
\draw[->] ($(api1)+(0.3,-0.2)$) -- node[near start,below,font=\scriptsize]{dados JSON} ($(nb1)-(0.5,0)$);
\node[cloud,draw,fill=white,cloud puffs=10,cloud puff arc=120,aspect=2,inner ysep=0.5em,right=2.5cm of api1,font=\scriptsize] (cl1) {Internet};
\end{tikzpicture}
\caption{Diagrama dos dados lidos pela API e enviados à aplicação}
\label{fig:api}
\end{figure}

\section{Biblioteca para tratamento de nomes}
O bolsista trabalhou em uma biblioteca de reconhecimento de nomes de pessoas baseado nos nomes disponíveis publicamente pelo site da Receita Federal. A primeira parte do trabalho envolveu a criação de uma tabela com todos os nomes e sobrenomes encontrados nos dados coletados. Em seguida foram criados três algoritmos para a análise de nomes.

O primeiro algoritmo é responsável por localizar um nome específico em um texto e analisar se o nome encontrado no texto é exatamente igual ao nome procurado ou se é maior, contendo um segundo sobrenome, por exemplo. O segundo algoritmo traz todos os nomes encontrados a partir de parte do nome e o terceiro traz todos os nomes encontrados em um texto.

A biblioteca desenvolvida poderá ser utilizada pelos robôs na análise de documentos texto, planilhas e página web.

\begin{figure}[h!]
\centering
\begin{tikzpicture}[>=stealth]
\node[database] (db3) at (0,0) {Receita Federal};
\node[database, right=1.5cm of db3] (db4) {Banco de nomes};
\node[api, right=1.5cm of db4] (api2) {L I B};
\node[robot, right=1.5cm of api2, label={[font=\scriptsize,align=center]south:robôs},yshift=-2mm] (rb1) {};
\node[robot, above=-1mm of rb1] (rb2) {};
\node[docs, right=1cm of rb1,yshift=2mm,label={[font=\scriptsize,align=center,yshift=2mm]south:documentos}] (doc1) {};
\draw[->] (db3) -- node[midway,above,font=\scriptsize]{dados} node[midway,below,font=\scriptsize]{tratados} (db4);
\draw[->] (db4) -- node[midway,above,font=\scriptsize]{nomes} (api2);
\draw[->] ($(api2)+(3mm,-2mm)$) -- node[midway,above,font=\scriptsize,sloped]{} ($(rb1)-(4mm,0)$);
\draw[->] ($(api2)+(3mm,4mm)$) -- node[midway,above,font=\scriptsize,sloped]{consulta} ($(rb2)-(4mm,4mm)$);
\draw[<-] ($(rb1)+(4mm,0)$) -- node[midway,above,font=\scriptsize]{} ($(doc1)-(7mm,2mm)$);
\draw[<-] ($(rb2)+(4mm,-4mm)$) -- node[midway,above,font=\scriptsize,sloped]{análise} ($(doc1)-(7mm,-4mm)$);
\end{tikzpicture}
\caption{Biblioteca de nomes auxilia na busca dos robôs}
\label{fig:biblioteca}
\end{figure}

\section{Refatoração dos robôs}
Aqui vem.

\chapter{Conclusão}
\label{chp:conclusao}
O bolsista participou da criação de um novo software, que representa um esforço para a análise
automática de imagens de retina para a deteção da RD. O trabalho do bolsista ajudou a consolidar
colaborações, divulgar e popularizar o método de segmentação e softwares através de publicações
e apoio a usuários ocasionais que demonstravam interesse. Essas contribuições são importantes
para que o software e a linha de pesquisa se desenvolvam, consolidando um ambiente propicio para
trabalhos futuros, que conta com colaboradores, um software em desenvolvimento, e o ambiente
da Incubadora Virtual da FAPESP.  O trabalho realizado ajudou a formar o bolsista, ensinando na
prática as atividades das fases de criação e manutenção de um software em ambiente colaborativo,
incluindo interação com usuários finais.

A \fomento\ financiou até o momento 4 meses do projeto, previsto para 17 meses, com a Bolsa Concedida como Itens Orçamentários em Auxílios (BCO) de Treinamento Técnico V (TT-V). O apoio financeiro da \fomentoSigla\ tem sido de considerável importância para a realização do projeto de pesquisa.

A empresa \instituicaoSigla\ acolhe a pesquisa e disponibiliza as condições necessárias para
sua realização com infraestrutura, algoritmos, acesso a Internet, dados coletados e orientação por parte do pesquisador \responsavel , responsável pela pesquisa.

\chapter{Atividades para o próximo período}
\label{chp:atividades}
Aqui começa o terceiro capítulo com a descrição \cite{postgresql} \cite{python} \cite{drf} da participação em eventos científicos.

\bibliographystyle{abntex2-num}
\bibliography{bibliografia}

\end{document}
